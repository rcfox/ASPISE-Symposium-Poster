\documentclass[landscape,final]{baposter}

\usepackage{times}
\usepackage{calc}
\usepackage{graphicx}
\usepackage{amsmath}
\usepackage{amssymb}
\usepackage{relsize}
\usepackage{multirow}
\usepackage{tikz}
\usepackage{bm}
\usepackage{color}
\usepackage{graphicx}
\usepackage{multicol}
\usepackage{verbatim}
\usepackage{pgfbaselayers}
\pgfdeclarelayer{background}
\pgfdeclarelayer{foreground}
\pgfsetlayers{background,main,foreground}

\usepackage{helvet}
%\usepackage{bookman}
\usepackage{palatino}

\newcommand{\captionfont}{\footnotesize}

\selectcolormodel{cmyk}

\graphicspath{{images/}}

%%%%%%%%%%%%%%%%%%%%%%%%%%%%%%%%%%%%%%%%%%%%%%%%%%%%%%%%%%%%%%%%%%%%%%%%%%%%%%%%
%%%% Some math symbols used in the text
%%%%%%%%%%%%%%%%%%%%%%%%%%%%%%%%%%%%%%%%%%%%%%%%%%%%%%%%%%%%%%%%%%%%%%%%%%%%%%%%
% Format 
\newcommand{\Matrix}[1]{\begin{bmatrix} #1 \end{bmatrix}}
\newcommand{\Vector}[1]{\Matrix{#1}}
\newcommand*{\SET}[1]  {\ensuremath{\mathcal{#1}}}
\newcommand*{\MAT}[1]  {\ensuremath{\mathbf{#1}}}
\newcommand*{\VEC}[1]  {\ensuremath{\bm{#1}}}
\newcommand*{\CONST}[1]{\ensuremath{\mathit{#1}}}
\newcommand*{\norm}[1]{\mathopen\| #1 \mathclose\|}% use instead of $\|x\|$
\newcommand*{\abs}[1]{\mathopen| #1 \mathclose|}% use instead of $\|x\|$
\newcommand*{\absLR}[1]{\left| #1 \right|}% use instead of $\|x\|$

\def\norm#1{\mathopen\| #1 \mathclose\|}% use instead of $\|x\|$
\newcommand{\normLR}[1]{\left\| #1 \right\|}% use instead of $\|x\|$

%%%%%%%%%%%%%%%%%%%%%%%%%%%%%%%%%%%%%%%%%%%%%%%%%%%%%%%%%%%%%%%%%%%%%%%%%%%%%%%%
% Multicol Settings
%%%%%%%%%%%%%%%%%%%%%%%%%%%%%%%%%%%%%%%%%%%%%%%%%%%%%%%%%%%%%%%%%%%%%%%%%%%%%%%%
\setlength{\columnsep}{0.7em}
\setlength{\columnseprule}{0mm}


%%%%%%%%%%%%%%%%%%%%%%%%%%%%%%%%%%%%%%%%%%%%%%%%%%%%%%%%%%%%%%%%%%%%%%%%%%%%%%%%
% Save space in lists. Use this after the opening of the list
%%%%%%%%%%%%%%%%%%%%%%%%%%%%%%%%%%%%%%%%%%%%%%%%%%%%%%%%%%%%%%%%%%%%%%%%%%%%%%%%
\newcommand{\compresslist}{%
\setlength{\itemsep}{1pt}%
\setlength{\parskip}{0pt}%
\setlength{\parsep}{0pt}%
}

\renewenvironment{itemize}
  {\begin{list}{\labelitemi}{\leftmargin=1em} \compresslist}
  {\end{list}}


%%%%%%%%%%%%%%%%%%%%%%%%%%%%%%%%%%%%%%%%%%%%%%%%%%%%%%%%%%%%%%%%%%%%%%%%%%%%%%
%%% Begin of Document
%%%%%%%%%%%%%%%%%%%%%%%%%%%%%%%%%%%%%%%%%%%%%%%%%%%%%%%%%%%%%%%%%%%%%%%%%%%%%%

\begin{document}

%%%%%%%%%%%%%%%%%%%%%%%%%%%%%%%%%%%%%%%%%%%%%%%%%%%%%%%%%%%%%%%%%%%%%%%%%%%%%%
%%% Here starts the poster
%%%---------------------------------------------------------------------------
%%% Format it to your taste with the options
%%%%%%%%%%%%%%%%%%%%%%%%%%%%%%%%%%%%%%%%%%%%%%%%%%%%%%%%%%%%%%%%%%%%%%%%%%%%%%
\typeout{Poster Starts}
\background{
  \begin{tikzpicture}[remember picture,overlay]%
%    \draw (current page.north west)+(-2em,-0em) node[anchor=north west] {\hspace{-2em}\includegraphics[height=1.1\textheight]{silhouettes_background}};
  \end{tikzpicture}%
}
\definecolor{silver}{cmyk}{0,0,0,0.3}
\definecolor{yellow}{cmyk}{0,0,0.9,0.0}
\definecolor{reddishyellow}{cmyk}{0,0.22,1.0,0.0}
\definecolor{black}{cmyk}{0,0,0.0,1.0}
\definecolor{darkYellow}{cmyk}{0,0,1.0,0.5}
\definecolor{darkSilver}{cmyk}{0,0,0,0.1}

\definecolor{lightyellow}{cmyk}{0,0,0.3,0.0}
\definecolor{lighteryellow}{cmyk}{0,0,0.1,0.0}
\definecolor{lighteryellow}{cmyk}{0,0,0.1,0.0}
\definecolor{lightestyellow}{cmyk}{0,0,0.05,0.0}
\begin{poster}{
  % Show grid to help with alignment
  grid=no,
  % Column spacing
  colspacing=1em,
  % Color style
  bgColorOne=lighteryellow,
  bgColorTwo=lightestyellow,
  borderColor=reddishyellow,
  headerColorOne=yellow,
  headerColorTwo=reddishyellow,
  headerFontColor=black,
  boxColorOne=lightyellow,
  boxColorTwo=lighteryellow,
  % Format of textbox
  textborder=roundedleft,
  % Format of text header
  eyecatcher=no,
  headerborder=open,
  headerheight=0.08\textheight,
  headershape=roundedright,
  headershade=plain,
  headerfont=\Large\textsf, %Sans Serif
  boxshade=plain,
%  background=shade-tb,
  background=plain,
  linewidth=2pt
  }
  % Eye Catcher
  {} % No eye catcher for this poster. If an eye catcher is present, the title is centered between eye-catcher and logo.
  % Title
  {\sf %Sans Serif
  %\bf% Serif
    Application-Specific Processor Instruction Set Extensions
  }
  % Authors
  {\sf %Sans Serif
  % Serif
    Ryan Fox, Peter Kotwicz, Anton Krutiansky, Byoung-Gi Lee % Alphabetical order
    \hspace{3em}
    University of Waterloo
  }
  % University logo
  {{\begin{minipage}{16em}
    \hfill
%    \includegraphics[height=2em]{msrlogo}
%    \includegraphics[height=5.5em]{logo}
  \end{minipage}}
  }

  \tikzstyle{light shaded}=[top color=baposterBGtwo!30!white,bottom color=baposterBGone!30!white,shading=axis,shading angle=30]

  % Width of left inset image
     \newlength{\leftimgwidth}
     \setlength{\leftimgwidth}{0.78em+8.0em}

%%%%%%%%%%%%%%%%%%%%%%%%%%%%%%%%%%%%%%%%%%%%%%%%%%%%%%%%%%%%%%%%%%%%%%%%%%%%%%
%%% Now define the boxes that make up the poster
%%%---------------------------------------------------------------------------
%%% Each box has a name and can be placed absolutely or relatively.
%%% The only inconvenience is that you can only specify a relative position 
%%% towards an already declared box. So if you have a box attached to the 
%%% bottom, one to the top and a third one which should be in between, you 
%%% have to specify the top and bottom boxes before you specify the middle 
%%% box.
%%%%%%%%%%%%%%%%%%%%%%%%%%%%%%%%%%%%%%%%%%%%%%%%%%%%%%%%%%%%%%%%%%%%%%%%%%%%%%
    %
    % A coloured circle useful as a bullet with an adjustably strong filling
    \newcommand{\colouredcircle}[1]{%
      \tikz{\useasboundingbox (-0.2em,-0.32em) rectangle(0.2em,0.32em); \draw[draw=black,fill=baposterBGone!80!black!#1!white,line width=0.03em] (0,0) circle(0.18em);}}

%%%%%%%%%%%%%%%%%%%%%%%%%%%%%%%%%%%%%%%%%%%%%%%%%%%%%%%%%%%%%%%%%%%%%%%%%%%%%%
  \headerbox{Overview}{name=overview,column=0,row=0}{
%%%%%%%%%%%%%%%%%%%%%%%%%%%%%%%%%%%%%%%%%%%%%%%%%%%%%%%%%%%%%%%%%%%%%%%%%%%%%%
    \begin{itemize}
    \item Problems:
      \begin{itemize}
      \item It's getting harder to make computers faster.
      \item Writing software is much easier than designing a better processor.
      \end{itemize}
    \item Solution:
      \begin{itemize}
      \item Optimize your hardware for your software!
      \end{itemize}
    \item How?
      \begin{itemize}
      \item Perform dynamic code analysis to find common patterns of instructions.
      \item Turn profitable patterns into new instructions.
      \end{itemize}
    \item Useful for:
      \begin{itemize}
      \item Creating CPU prototypes.
      \item Embedded system running a limited range of applications.
      \item Design of GPU by identifying potential new instructions.
      \item Optimization of general programs by pattern finding.
      \end{itemize}
   \end{itemize}
  }


%% %%%%%%%%%%%%%%%%%%%%%%%%%%%%%%%%%%%%%%%%%%%%%%%%%%%%%%%%%%%%%%%%%%%%%%%%%%%%%%
%%   \headerbox{Applications}{name=applications,column=1,span=2,row=0}{
%% %%%%%%%%%%%%%%%%%%%%%%%%%%%%%%%%%%%%%%%%%%%%%%%%%%%%%%%%%%%%%%%%%%%%%%%%%%%%%%
    
%%  }


%%%%%%%%%%%%%%%%%%%%%%%%%%%%%%%%%%%%%%%%%%%%%%%%%%%%%%%%%%%%%%%%%%%%%%%%%%%%%%
  \headerbox{Definitions}{name=definitions,column=3,row=0}{
    %%%%%%%%%%%%%%%%%%%%%%%%%%%%%%%%%%%%%%%%%%%%%%%%%%%%%%%%%%%%%%%%%%%%%%%%%%%%%%
    \begin{itemize}
      \item \textbf{ASPISE} Application-Specific Processor Instruction Set Extensions
      \item \textbf{CPI} Cycles Per Instruction
      \item \textbf{SIMD} Single Instruction, Multiple Data
      \item \textbf{GPU} Graphics Processing Unit
      \item \textbf{RTL} Register Transfer Language
      \item \textbf{FPGA} Field-Programmable Gate Array
   \end{itemize}
 }

%%%%%%%%%%%%%%%%%%%%%%%%%%%%%%%%%%%%%%%%%%%%%%%%%%%%%%%%%%%%%%%%%%%%%%%%%%%%%%
  \headerbox{Targeted Application}{name=target,below=overview,above=bottom}{
%%%%%%%%%%%%%%%%%%%%%%%%%%%%%%%%%%%%%%%%%%%%%%%%%%%%%%%%%%%%%%%%%%%%%%%%%%%%%%
    \begin{itemize}
    \item Edge Detection
      \begin{itemize}
      \item Uses the Kirsch edge detection algorithm to find the edges in an image.
      \item Commonly used for computer vision.
      \item Many repetitions over loops.
      \begin{itemize}
        \item More chance for optimization.
      \end{itemize}
      \item Speed is important!
      \end{itemize}
    \item Running on SimpleScalar simulating ARM7.
    \end{itemize}

    \begin{center}
      \begin{tabular}{c c}
        Input & Output \\
        \includegraphics[scale=0.75]{medium} &
        \includegraphics[scale=0.75]{medium_out}  
      \end{tabular}
    \end{center}    
  }
  % Define block styles
  \tikzstyle{block} = [rectangle, draw, fill=blue!20, 
  text width=5em, text centered, rounded corners, minimum height=4em]
  \tikzstyle{line} = [draw, -latex]

  %%%%%%%%%%%%%%%%%%%%%%%%%%%%%%%%%%%%%%%%%%%%%%%%%%%%%%%%%%%%%%%%%%%%%%%%%%%%%%
  \headerbox{Impementation}{name=implementation,column=1,span=2,row=0}{
    %%%%%%%%%%%%%%%%%%%%%%%%%%%%%%%%%%%%%%%%%%%%%%%%%%%%%%%%%%%%%%%%%%%%%%%%%%%%%%

    \includegraphics[scale=0.4]{block_diagram}
}

  %%%%%%%%%%%%%%%%%%%%%%%%%%%%%%%%%%%%%%%%%%%%%%%%%%%%%%%%%%%%%%%%%%%%%%%%%%%%%%
  \headerbox{Pattern Finding}{name=patternfinding,column=1,span=2,below=implementation}{
    %%%%%%%%%%%%%%%%%%%%%%%%%%%%%%%%%%%%%%%%%%%%%%%%%%%%%%%%%%%%%%%%%%%%%%%%%%%%%%
    {\ttfamily
    ldr \textbf{\color{red}ldr add} add mov mov str \textbf{\color{red}ldr add} str \textbf{\color{red}ldr add} str ldr ldr cmp
    blt ldr \\
    \textbf{\color{red}ldr add} mov mov mov ldr \textbf{\color{red}ldr add} add mov mov str mov ldr \textbf{\color{red}ldr add}
    add mov \\
    mov str \textbf{\color{red}ldr add} str \textbf{\color{red}ldr add} str ldr ldr cmp blt ldr \textbf{\color{red}ldr add} mov
    mov str \\
    mov ldr \textbf{\color{red}ldr add} add mov mov str mov ldr \textbf{\color{red}ldr add} add mov mov str
    \textbf{\color{red}ldr add} \\
    str \textbf{\color{red}ldr add} str ldr ldr cmp blt ldr \textbf{\color{red}ldr add} mov mov str mov ldr
    \textbf{\color{red}ldr add} \\
    add mov mov str mov ldr \textbf{\color{red}ldr add} add mov add add mov mov str mov
    ldr ldr \\
    }

    \begin{itemize}
      \item Some patterns occur more frequently than others and thus
        are more profitable.
      \item Finding patterns that provide the best optimization is
        difficult but worthwhile.
    \end{itemize}
  }

  %%%%%%%%%%%%%%%%%%%%%%%%%%%%%%%%%%%%%%%%%%%%%%%%%%%%%%%%%%%%%%%%%%%%%%%%%%%%%%
  \headerbox{Instruction Generation}{name=generation,column=1,span=2,below=patternfinding,
    above=bottom} {
    %%%%%%%%%%%%%%%%%%%%%%%%%%%%%%%%%%%%%%%%%%%%%%%%%%%%%%%%%%%%%%%%%%%%%%%%%%%%%%
    \begin{center}
      \includegraphics[trim=25mm 35mm 18mm 60mm,clip=true,scale=0.5]{arm_instructions}
    \end{center}

    \begin{itemize}
    \item Take advantage of undefined opcode range for new instructions.
    \item Generating new instructions incorporates many constraints from
      CPU, memory, and old instructions.
    \item Constraints in bus bandwidth, instruction length and size/number of operands.
   \end{itemize}
    }

%%%%%%%%%%%%%%%%%%%%%%%%%%%%%%%%%%%%%%%%%%%%%%%%%%%%%%%%%%%%%%%%%%%%%%%%%%%%%%
  \headerbox{Performance Analysis}{name=analysis,column=3,below=definitions}{
%%%%%%%%%%%%%%%%%%%%%%%%%%%%%%%%%%%%%%%%%%%%%%%%%%%%%%%%%%%%%%%%%%%%%%%%%%%%%%
    \begin{itemize}
      \item Instruction candidates obtained during generation.
    \end{itemize}

    \begin{tabular}{| l | c | c | c |}
      \hline
      \textbf{inst group} & \textbf{counts} & \textbf{CPI} & \textbf{Speedup(\%)} \\ \hline
      ldr add ldr & 12692 & 7 & 10.27 \\ \hline
      add ldr & 21704 & 4 & 10.21 \\ \hline
      ldr add & 20307 & 4 & 9.30 \\ \hline
      ldr add ldr add & 9700 & 8 & 8.85 \\ \hline
    \end{tabular}

    \begin{itemize}
      \item Run a test program on original and extended ARM7 and
        compare results
      \item Kirsch edge detector with $500  \times 333$ image.
      \item \texttt{ldr add ldr} combined into one instruction.
    \end{itemize}

    \begin{tabular}{| l | c | c | c |}
      \hline
      Desc & Original & New & \% diff \\ \hline
      Total Cycles & 64649614 & 63698642 & -1.49 \\ \hline
      CPI & 0.5672 & 0.5931 & 4.37 \\ \hline
      Speed (insts/s) & 340268 & 309495 & -9.94 \\ \hline
     % \cline{2-3}
      % & \multicolumn{2}{|c|}{Bitmap Size} \\ \cline{2-3}
      % & $20 \times 20$ & $100 \times 100$ \\ \hline
      % Total cycles & 477166 & 31564966 \\ \hline
      % Pattern occurrences & 18307 & 1211324 \\ \hline
      % inst cycles & 4 & 4 \\ \hline
      % new inst cycles & 3 & 3 \\ \hline
      % \textbf{Speedup} & 4\% & 4\% \\ \hline
    \end{tabular}

    \begin{itemize}
      \item Most performance gain is from increased instruction speed.
      \item It is difficult to reach beyond a certain point due to
        various contraints.
   \end{itemize}
  }

%%%%%%%%%%%%%%%%%%%%%%%%%%%%%%%%%%%%%%%%%%%%%%%%%%%%%%%%%%%%%%%%%%%%%%%%%%%%%%
  \headerbox{Future Research}{name=research,column=3,below=analysis, above=bottom}{
%%%%%%%%%%%%%%%%%%%%%%%%%%%%%%%%%%%%%%%%%%%%%%%%%%%%%%%%%%%%%%%%%%%%%%%%%%%%%%
    \begin{itemize}
      \item Automate all stages from benchmarking to instruction generation.
      \item Just-In-Time instruction generation on each run.
      \item Synthesize FPGA with extended instruction set.
      \item Instruction generation using \texttt{gcc} RTL representation.
    \end{itemize}
   }

\end{poster}%
%
\end{document}
